\documentclass[a4paper]{{article}}
\usepackage[numbers]{{natbib}}
\usepackage[english]{{babel}}
\usepackage[utf8]{{inputenc}}
\usepackage{{amsmath}}
\usepackage{{graphicx}}
\usepackage{{xcolor}}
\usepackage{{authblk}}
\usepackage{{caption}}
\usepackage{{subcaption}}
\usepackage{{hyperref}}
\usepackage[outdir=./converted]{{epstopdf}}
\usepackage[left=30mm, right=30mm, bottom=30mm]{{geometry}}
\usepackage{{fancyhdr}}
\fancypagestyle{{firstpage}}{{
\renewcommand{{\headrulewidth}}{{0pt}}
\fancyhead[L]{{
\includegraphics[width=3cm]{{/scratch9/tglauch/realtime_service/main/latex/asi_logo.jpg}}
}}
\fancyhead[R]{{
\includegraphics[width=3cm]{{/scratch9/tglauch/realtime_service/main/latex/ias_logo.jpg}}
}}
}}
\title{{Dissecting the Region around \textit{{{event}}}}}

\author[1,3,$\dagger$]{{Theo Glauch}}
\author[2,3]{{Paolo Giommi}}
\affil[1]{{Technical University Munich, Garching, Germany}}
\affil[2]{{Agenzia Spaziale Italiana, ASI, via del Politecnico s.n.c., I-00133 Roma Italy}}
\affil[3]{{Institute for Advanced Studies, Technical University Munich, Garching, Germany}}
\affil[$\dagger$]{{theo.glauch@tum.de}}
\date{{\today}}

\begin{{document}}
\setlength{{\headheight}}{{80pt}}
\maketitle
\thispagestyle{{firstpage}}
\begin{{abstract}}
{prelim}The IceCube Neutrino Observatory has recently reported the detection of a highly-energetic
neutrino, \textit{{{event}}} \footnote{{\protect\url{{{gcnurl}}}}} (MJD: {date:.2f}), with direction ra: {ra:.2f}$^\circ$, dec: {dec:.2f}$^\circ$ in equatorial and l: {l:.2f}$^\circ$, b:
{b:.2f}$^\circ$ in galactic coordinates. Due to the high-energy and the track-like signature the event has a good
pointing and is likely of astrophysical origin. With the goal of identifing
the corresponding electromagnetic counterpart we report here on the muli-wavenlength dissection of the region around the
event. This report was generated autmoatically on {createdon} UTC.
\end{{abstract}}

\section{{Identifying Counterpart Candidates}}
The search for possible counterparts is based on the \textit{{VOU-Blazars}} tool \cite{{voublazar}} and follows closely the pipeline
developed in \cite{{Padovani:2018acg}}. \textit{{VOU-Blazars}} compares 32 multi-wavelength catalogs\footnote{{List of the 32 catalogs used in this
analysis: SDS82 , 3HSP , Fermi8YR ,1BIGB, MST9Y, FIRST, SUMSS, WGACAT, IPC2E, ZWCLUSTERS,
PSZ2, ABELL , SDSSWHL, CRATES, NVSS, SXPS, RASS, XMMSL, BMW , IPCSL, CHANDRA, MCXC,  5BZCat, SWXCS, PULSAR, F2PSR,
3FHL, 3FGL, 3XMM, MAXI, FermiMEV, AGILE}}
to find all positions in the vicinity of the alert with a \textit{{blazar-like}} emission profile in radio, optical and
X-ray. Each of these matches is then additionally checked against existing catalogs to identify the associated object if
possible. \\

In a second step a dedicated gamma-ray analysis is performed. In order to search for interesting emission features in the region we use \textit{{Fermi LAT}} data around the event time and run three different analysis
pipelines. Firstly, test-statistic maps are generated to search for unknown gamma-ray emmitters, e.g. indication for
gamma-ray emission from the previously identfied \textit{{VOU-Blazars}} source candidates. Subsequently SEDs and light curves are produced for each identified catalog
source and interesting \textit{{VOU-Blazars}} candidate. Based on these results we can finally search for specific
features in flux and spectral shape around the neutrino arrival time. \\

The gamma-ray analyses are based on the latest version of the FSSC Tools v11r5p3 and the fermipy package
\cite{{fermipy}}. We use the the standart procedures as described in the \textit{{Fermi LAT}} Cicerone \cite{{fermi_ci}}. The
dataset for the analysis contains events with photon energy above {emin} GeV in a time window between {mjd1:.1f} and {mjd2:.1f}.
\newpage
\setlength{{\headheight}}{{0mm}}
\thispagestyle{{plain}}
\subsection{{Catalog Sources in the Region around {event}}}
The following list gives a quick overview about the catalog blazars in the region. Sources are sorted by their angular distance to the neutrino direction.
\\ \\
{cat_srcs}

\section{{Full Multi-Wavelength Study of the Region}}
\subsection{{Description of the Analysis}}
In this section we present a full multi-wavelength search for possible neutrino counterparts. Starting from 32
multi-wavenlength catalogs the VOU-Blazar tool \cite{{voublazar}} uses the all the available radio, optical and X-ray
data in order to identify blazar-like counterparts candidates. The full output of the tool cand be found in the appendix.\\

The analysis pipeline consists of two parts: 1) The radio and x-ray data, as well as the resulting counterpart candidates are shown
and compared to the \textit{{Fermi LAT}} gamma-ray emission in the region around the neutrino alert 2) For all known blazars with a angular distance of less
than 1.5 degrees a multi-wavelength SED is constructed, including also a gamma-ray analysis that is started at the time of the
neutrino alert. For each source we also calculated a fixed-binning light curve. For details about the analysis see the
appendix. \\

\subsection{{The Multi-Wavelength SED}}
\label{{sec:sed}}
The multi-wavelength SED collects and visualizes all the publicly available multi-wavelength data, as well as the result of the
gamma-ray analysis. The time evolution of the source is decoded in a color gradient from grey (old) to red (recent).
Here the grey SEDs point and bowties represent the \textit{{Fermi-LAT}} gamma-ray spectrum integrated over the entire
mission while the black SED points show the gamma-ray spectrum in a time window around the
neutrino arrival time. Colored bands indicate the corresponing spectral fits at different (if available) energy
thresholds if the significance of the measurement is above 3 $\sigma$. The green dashed and solid line show the sensitivity and discovery potential of the IceCube 7yr point-source analysis \cite{{psana}}, respectively.

\clearpage
\begin{{figure}}[h!]
\centering
\begin{{subfigure}}{{.45\textwidth}}
  \centering
  \includegraphics[width=.8\linewidth]{{{rx_map}}}
\end{{subfigure}}%
\begin{{subfigure}}{{.55\textwidth}}
\centering
  \includegraphics[width=.95\linewidth]{{{vou_pic}}}
\end{{subfigure}}
 \caption{{Left Plot: Radio and X-ray sources within 120 arc-minutes of the position of the neutrino event. Symbol
diameters are proportional to source intensity. Radio sources appear as red filled circles, X-ray sources as open blue
circles and gamma-ray sources as open triangles. Right Plot: Counterpart candidates in a 120 arc-minutes radius around the event direction. Dark blue circles represent LBL type candidates, that is sources with flux ratio in the range observed in the sample of LBL blazars of
the latest edition of the BZCAT catalogue \cite{{Massaro:2015nia}}, cyan symbols are for IBL type candidates, and orange symbols are for
HBL candidates. Known blazars are marked by a red diamond if they are included in the BZCAT catalogue or a star if
they are part of the 2WHSP sample. The shaded area marks a circle of 90 arcmis around the events-best fit direction.}}
\end{{figure}}


\begin{{figure}}[h!]
\centering
\begin{{subfigure}}{{.38\textwidth}}
  \includegraphics[width=\textwidth]{{{ts_map_short}}}
\end{{subfigure}}%
\begin{{subfigure}}{{.38\textwidth}}
\centering
  \includegraphics[width=\textwidth]{{{ts_map}}}
\end{{subfigure}}%
\begin{{subfigure}}{{.19\textwidth}}
\centering
  \includegraphics[width=\textwidth]{{{ts_map_legend}}}
\end{{subfigure}}
\caption{{Test-statistic maps of the region after substracting know sources from the 4FGL catalog. The blue and
black contours show the 90\% and 50\% error regions, respectively. Only photons with energies above {tsemin} GeV are
included. The map can be used to
identify additional, yet unkown gamma-ray emitters, that coincide with multi-wavlength candidates. Contour lines are
shown at 2,3,4,5 $\sigma$. \textbf{{Left}}: The map of the region for a time window of 200 days (MJD {tsmjd1:.1f}, {tsmjd2:.1f}) around the
neutrino arrival time. \textbf{{Right}}: The map of
the region for the entire \textit{{Fermi-LAT}} mission (MJD {mjd1:.1f} to {mjd2:.1f}).}}
\end{{figure}}

\clearpage
\section{{SEDs and Light Curves}}
\label{{source_sum}}
{src_latex}

\clearpage
\section{{Appendix}}
\subsection{{Analyis Details - Light Curve}}
\subsubsection{{Ligh Curve Binning}}
Since we want to avoid running an extremly time consuming adaptive binning algorithm for all sources we try to estimate
reasonable time windows for each
source based on the following criteria:
\begin{{itemize}}
\item a) If possible we want to have a significant detection in each time bin
\item b) IceCube needs on an order of $\mathcal{{O}}$(100 days) integration time to detect a significant neutrino singal in a
point-source analysis
\item c) following from b) we don't care too much about extremly short time-windows, since they are experimental hard to
access
\item d) unkown gamma-ray emitters are not expected to have a large time-integrated signal, but can still have gamma-ray
outburst/flares
\end{{itemize}}
As a result of this we adapt the following procedure:
\begin{{itemize}}
\item For unkown gamma-ray emitters we built a light curve with time windows of 100 days, starting from the time of the
neutrino alert. In case of subsequent measurements close to the detection threshold, an adaptive binning light curve can
be run as a follow-up
\item For known gamma-ray emitters, i.e. sources listed in the 4FGL catalog, we calculate the time window need for a
$5\sigma$ ($3 \sigma$) detection assuming a constant emission. This value is then used as time window, with the only
limitation that we limit it to maximal 200 days in order to avoid missing interesting fluctuations around the neutrino
arrival time.
\end{{itemize}}
The calculation of the time-windows for known-gamma ray emitters is based on the asymptotic behaviour of counting
experiments combined with the information given in the 4FGL catalog. For a counting experiments with $\chi^{{2}}$
background test-statistic distribution in the asymptotic limit the median test statistic value of a signal behaves as
\cite{{Cowan:2010js}}
\begin{{equation}}
\mathcal{{TS}} = 2 \times\left[(s+b)\ln\left(1+\frac{{s}}{{b}}\right)-s\right].
\end{{equation}}
As Fermi-LAT has only limited background we consider this equation in the limit of $s>>b>>1$ and $s+b \rightarrow
\infty$, hence we can simplify to
\begin{{equation}}
\mathcal{{TS}} \rightarrow 2\times s\,\left [\ln\left(\frac{{s}}{{b}}\right)\right]
\end{{equation}}
rewriting $s = s_0t$ and $b=b_0t$ we finally get
\begin{{equation}}
\mathcal{{TS}} \rightarrow 2\times s_0t\, \left[\ln\left(\frac{{s_0}}{{b_0}}\right)\right]
\label{{eq:final}}
\end{{equation}}
which scales linearly in time. The 4FGL catalog provides the 8 year time-integrated test-statistic values for each
source in 7
energy bands. In order to calculate the total test-statistic value for our selection we sum up all the test-statistic
values with energies larger than the minimimum (threshold) energy of this analysis. Finally we can calculate the
integration time needed for $5\sigma$ ($\mathcal{{TS}} = 25$) and $3\sigma$ ($\mathcal{{TS}} = 9$) detection using
equation \eqref{{eq:final}}
\begin{{equation}}
t(5\sigma) = \frac{{25}}{{\mathcal{{TS}}_{{8years}}}}\cdot 2920\,[\mathrm{{days}}]
\end{{equation}}

\subsection{{Analyis Details - SED}}
The construction of the SED of the counterpart candiates is based on the \textit{{VOU-Blazars}} tool \cite{{voublazar}}.
For a given source position it identifies all the corresponding measurements based on likelihood ratios test taking into
account the specific point spread functions of the 32 respective experiments/catalogs taken into account in our
analysis \footnote{{List of the 32 catalogs used in this
analysis: SDS82 , 3HSP , Fermi8YR ,1BIGB, MST9Y, FIRST, SUMSS, WGACAT, IPC2E, ZWCLUSTERS,
PSZ2, ABELL , SDSSWHL, CRATES, NVSS, SXPS, RASS, XMMSL, BMW , IPCSL, CHANDRA, MCXC,  5BZCat, SWXCS, PULSAR, F2PSR,
3FHL, 3FGL, 3XMM, MAXI, FermiMEV, AGILE}}. This procedure works stable for multi-wavelength catalogs, as well as single band measurements except for
gamma-ray catalogs where the point-spread function is comparably large. For the construction of the SED of Fermi-LAT
counterparts the procedure is hence two-step. 1) We run a counterpart search in the given 95\% confidence region of the
source position. In most cases there is only one (or no) possible multi-wavelength counterpart candidate, in all other cases we
assume the strongest source to be the countepart. 2) We construct the SED using the location of the indentified
counterpart. In all of these cases the skymap in the vicinity with all the possible multi-wavelength counterparts is
additionally shown in the source summary in section \ref{{source_sum}}. Finally the IceCube point source sensitivity and
discovery potential are taken from \cite{{psana}} and shown for reference.

\clearpage
\subsection{{Full VOU Output}}
{vou_output}


{{\small\bibliographystyle{{unsrt}}
\bibliography{{sample}}}}
\end{{document}}
