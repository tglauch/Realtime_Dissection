\documentclass[a4paper]{{article}}

\usepackage[english]{{babel}}
\usepackage[utf8]{{inputenc}}
\usepackage{{amsmath}}
\usepackage{{graphicx}}
\usepackage[left=30mm, right=30mm, bottom=30mm]{{geometry}}
\title{{Dissecting the Region around \textit{{{event}}}}}

\author{{Authors}}

\date{{\today}}

\begin{{document}}
\maketitle

\begin{{abstract}}
We report here on the muli-wavenlength dissection of the region around the IceCube alert event \textit{{{event}}}
\end{{abstract}}

\section{{Multi-Wavenlength Analysis of the Region around \textit{{{event}}}}}
The search for possible counterparts is based on the VOU-Blazar Tool \ref{{voublazar}}. The tool compares existing catalogs to find all positions in the region around the neutrino event that have a \textit{{blazar-like}} radio to X-ray emission. For each match gamma-ray catalogs are checked to identify possible counterparts. \\

The following gamma-ray analysis is based on the latest version of the FSSC Tools v11r5p3 [ref] and the fermipy package. The dataset contains photon energies above 2 GeV and a time window between [MJD1] and [MJD2]

\subsection{{Quick Summary: Catalog Sources in the Region}}
The following list gives a short overview about the most interesting catalog sources in the region. Check the following section for a more detailed analysis. The sources are sorted by their distance to the neutrino direction.
\\ \\
{cat_srcs}

\clearpage
\subsection{{Full MW Study of the Region}}
In this section the full multi-wavelength search for neutrino counterparts is presented. Despite catalog sources also candidate sources are listed based on the x-ray to radio emission. The sources are sorted by their right ascension.
\\ \\ \\
{vou_output}

\clearpage

\begin{{figure}}[h!]
\centering
  \includegraphics[width=1.0\textwidth]{{{vou_pic}}}
  \caption{{Radio and X-ray sources within 90 arc-minutes of the position of the event. Symbol diameters are proportional to source intensity. Radio sources appear as red filled circles, X-ray sources as open blue circles}}
\end{{figure}}

\begin{{figure}}[h!]
\centering
  \includegraphics[width=0.9\textwidth]{{{ts_map}}}
  \caption{{The TS Map of the region. The analysis model includes all the known sources from the 3FGL Catalog}}
\end{{figure}}

\begin{{figure}}[h!]
\centering
  \includegraphics[width=0.9\textwidth]{{{res_map}}}
  \caption{{The Residual Map of the region. The analysis model includes all the known sources from the 3FGL Catalog}}
\end{{figure}}

\clearpage
\section{{Sources SEDs and Lightcurves}}
{src_latex}


\begin{{thebibliography}}{{9}}
\bibitem{{voublazar}}
  Y.Chang,
  \emph{{https://github.com/ecylchang/VOU\_Blazars}}.
\end{{thebibliography}}
\end{{document}}
