\documentclass[a4paper]{{article}}
\usepackage[numbers]{{natbib}}
\usepackage[english]{{babel}}
\usepackage[utf8]{{inputenc}}
\usepackage{{amsmath}}
\usepackage{{graphicx}}
\usepackage{{xcolor}}
\usepackage{{authblk}}
\usepackage{{caption}}
\usepackage{{subcaption}}
\usepackage[outdir=./converted]{{epstopdf}}
\usepackage[left=30mm, right=30mm, bottom=30mm]{{geometry}}
\title{{Dissecting the Region around \textit{{{event}}}}}

\author[1,$\dagger$]{{Theo Glauch}}
\affil[1]{{Technical University Munich, Garching, Germany}}
\affil[$\dagger$]{{theo.glauch@tum.de}}
\date{{\today}}

\begin{{document}}
\maketitle

\begin{{abstract}}
{prelim}On MJD {date:.2f} the IceCube Neutrino Observatory  has reported the detection of a highly-energetic neutrino,
\textit{{{event}}}, with direction ra: {ra:.2f}$^\circ$, dec: {dec:.2f}$^\circ$ in equatorial and l: {l:.2f}$^\circ$, b:
{b:.2f}$^\circ$ in galactic coordinates. Due to the high-energy and the track-like signature the event has a good
pointing and is likely of astrophysical origin. With the goal of identifing
the corresponding electromagnetic counterpart we report here on the muli-wavenlength dissection of the region around the event.
\end{{abstract}}

\section{{Identifying counterpart candidates}}
The search for possible counterparts is based on the VOU-Blazar Tool \cite{{voublazar}} and follows closely the pipeline
developed in \cite{{Padovani:2018acg}}.  VOU-Blazar compares 32 multi-wavelengths catalogs\footnote{{List of the 32 catalogs used in this
analysis: SDS82 , 3HSP , Fermi8YR ,1BIGB, MST9Y, FIRST, SUMSS, WGACAT, IPC2E, ZWCLUSTERS,
PSZ2, ABELL , SDSSWHL, CRATES, NVSS, SXPS, RASS, XMMSL, BMW , IPCSL, CHANDRA, MCXC,  5BZCat, SWXCS, PULSAR, F2PSR,
3FHL, 3FGL, 3XMM, MAXI, FermiMEV, AGILE}}
to find all positions in the vicinity of the alert event with a \textit{{blazar-like}} radio
to X-ray emission ratio. Each match is then checked against existing blazar catalogs to identify possible associations. \\

In order to search for gamma-ray anomalies in the region we use Fermi-LAT data around the event time and run three different analysis
pipelines. Firstly, TS and residual maps are generated to search for unknown gamma-ray emmitters, e.g. indication for
emission from the previously identfied \textit{{VOU-Sources}} without gamma-ray association. At the same time the
residual map gives information of the 
the goodness of fit using the 3FGL model. Subsequently SEDs and light curves are produced for each identified catalog
source. This allows as to search for specific features in flux and spectral shape around the neutrino arrival time. The
analyses are based on the latest version of the FSSC Tools v11r5p3 and the fermipy package
\cite{{fermipy}}. We use the the standart procedures as described in the Fermi-LAT Cicerone \cite{{fermi_ci}}. The
dataset for the analysis contains events with photon energy above {emin} GeV in a time window between {mjd1:.1f} and {mjd2:.1f}.

\subsection{{Quick summary: Catalog sources in the region}}
The following list gives a quick overview about the catalog blazars and flat-spectrum radio sources in the region. The
full output of VOU-Blazar is given below. Sources are sorted by their angular distance to the neutrino direction.
\\ \\
{cat_srcs}

\subsection{{Full multi-wavelength study of the region}}
In this section we present the full multi-wavelength search for possible neutrino counterparts. Starting from 32
multi-wavenlength catalogs the VOU-Blazar Tool \cite{{voublazar}} uses the radio to x-ray emission ratio in order to search for
non-thermal counterparts candidates. In the following sources are sorted by their right ascension.
\\ \\ \\
{vou_output}

\clearpage
\begin{{figure}}[h!]
\centering
\begin{{subfigure}}{{.5\textwidth}}
  \centering
  \includegraphics[width=.8\linewidth]{{{rx_map}}}
\end{{subfigure}}%
\begin{{subfigure}}{{.5\textwidth}}
\centering
  \includegraphics[width=.8\linewidth]{{{vou_pic}}}
\end{{subfigure}}
 \caption{{Left Plot: Radio and X-ray sources within 120 arc-minutes of the position of the neutrino event. Symbol
diameters are proportional to source intensity. Radio sources appear as red filled circles, X-ray sources as open blue
circles  Right Plot: Counterpart candidates in a 120 arc-minutes radius around the event direction. Dark blue circles represent LBL type candidates, that is sources with flux ratio in the range observed in the sample of LBL blazars of
the latest edition of the BZCAT catalogue \cite{{Massaro:2015nia}}, cyan symbols are for IBL type candidates, and orange symbols are for
HBL candidates. Known blazars are marked by a red diamond if they are included in the BZCAT catalogue or a star if
they are part of the 2WHSP sample. The shaded area marks a circle of 90 arcmis around the events-best fit direction. }}
\end{{figure}}


\begin{{figure}}[h!]
\centering
\begin{{subfigure}}{{.37\textwidth}}
  \includegraphics[height=7cm]{{{ts_map_short}}}

\end{{subfigure}}%
\begin{{subfigure}}{{.63\textwidth}}
\centering
  \includegraphics[height=7cm]{{{ts_map}}}
\end{{subfigure}}
\caption{{Left: The TS map of the region for a time window of 200 days (MJD {tsmjd1:.1f}, {tsmjd2:.1f}) around the
neutrino arrival time. Right: The TS map of
the region for the entire Fermi LAT mission(MJD {mjd1:.1f} to {mjd2:.1f}). Both plots are down for energies above
{energy} GeV. The analysis model includes all known sources from the 3FGL catalog in a region of 8 degrees around the
center. The blue circle indicates a radius of 90 arcmin around the the event best-fit direction. }}
\end{{figure}}

\clearpage
\section{{The closer look: SEDs and light curves}}
{src_latex}

{{\small\bibliographystyle{{unsrt}}
\bibliography{{sample}}}}
\end{{document}}
