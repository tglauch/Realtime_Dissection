\documentclass[a4paper]{{article}}
\usepackage[numbers]{{natbib}}
\usepackage[english]{{babel}}
\usepackage[utf8]{{inputenc}}
\usepackage{{amsmath}}
\usepackage{{graphicx}}
\usepackage[left=30mm, right=30mm, bottom=30mm]{{geometry}}
\title{{Dissecting the Region around \textit{{{event}}}}}

\author{{Authors}}

\date{{\today}}

\begin{{document}}
\maketitle

\begin{{abstract}}
On !date! IceCube has reported the detection of an highly-energetic neutrino, \textit{{{event}}}, with direction
ra: {ra:.2f}$^\circ$, dec: {dec:.2f}$^\circ$ (l: {l:.2f}$^\circ$, b: {b:.2f}$^\circ$) that has a good chance of being of astropysical origin. With the goal of identifing
the corresponding electromagnetic  counterpart we report here on the muli-wavenlength dissection of the region around the event.
\end{{abstract}}

\section{{Multi-Wavenlength Analysis of the Region around \textit{{{event}}}}}
The search for possible counterparts is based on the VOU-Blazar Tool \cite{{voublazar}}. The tool compares existing
multi-wavelengths catalogs to find all positions in the region around the event that have a \textit{{blazar-like}} radio
to X-ray emission ratio. For each match existing gamma-ray catalogs are checked to identify possible counterparts. \\

The following gamma-ray analysis is based on the latest version of the FSSC Tools v11r5p3 and the fermipy package
\cite{{fermipy}}
\cite{{fermipy}}.
The dataset contains photon energies above {emin} GeV and a time window between {mjd1:.1f} and {mjd2:.1f}

\subsection{{Quick Summary: Catalog Sources in the Region}}
The following list gives a short overview about the most interesting catalog sources in the region. Check the following section for a more detailed analysis. The sources are sorted by their distance to the neutrino direction.
\\ \\
{cat_srcs}

\clearpage
\subsection{{Full MW Study of the Region}}
In this section the full multi-wavelength search for neutrino counterparts is presented. Despite catalog sources also candidate sources are listed based on the x-ray to radio emission. The sources are sorted by their right ascension.
\\ \\ \\
{vou_output}

\clearpage
\begin{{figure}}[h!]
\centering
  \includegraphics[width=0.8\textwidth]{{{rx_map}}}
  \caption{{Radio and X-ray sources within 120 arc-minutes of the position of the neutrino event. Symbol diameters are
proportional to source intensity. Radio sources appear as red filled circles, X-ray sources as open blue circles}}
\end{{figure}}

\clearpage

\begin{{figure}}[h!]
\centering
  \includegraphics[width=0.8\textwidth]{{{vou_pic}}}
  \caption{{Radio and X-ray sources within 120 arc-minutes of the position of the event. Symbol diameters are proportional to source intensity. Radio sources appear as red filled circles, X-ray sources as open blue circles}}
\end{{figure}}

\begin{{figure}}[h!]
\centering
  \includegraphics[width=0.9\textwidth]{{{ts_map}}}
  \caption{{The TS Map of the region for MJD {mjd1:.1f} and {mjd2:.1f} and energies above {energy} GeV. The analysis model includes all the known sources from the 3FGL Catalog}}
\end{{figure}}

\begin{{figure}}[h!]
\centering
  \includegraphics[width=0.9\textwidth]{{{res_map}}}
  \caption{{The Residual Map of the region for MJD {mjd1:.1f} and {mjd2:.1f} and energies above {energy} GeV. The analysis model includes all the known sources from the 3FGL Catalog}}
\end{{figure}}

\clearpage
\section{{Sources SEDs and Lightcurves}}
{src_latex}

{{\small\bibliographystyle{{unsrt}}
\bibliography{{sample}}}}
\end{{document}}
