\documentclass[a4paper]{{article}}
\usepackage[numbers]{{natbib}}
\usepackage[english]{{babel}}
\usepackage[utf8]{{inputenc}}
\usepackage{{amsmath}}
\usepackage{{graphicx}}
\usepackage{{xcolor}}
\usepackage{{authblk}}
\usepackage{{caption}}
\usepackage{{subcaption}}
\usepackage[outdir=./converted]{{epstopdf}}
\usepackage[left=30mm, right=30mm, bottom=30mm]{{geometry}}
\title{{Dissecting the Region around \textit{{{event}}}}}

\author[1,$\dagger$]{{Theo Glauch}}
\affil[1]{{Technical University Munich, Garching, Germany}}
\affil[$\dagger$]{{theo.glauch@tum.de}}
\date{{\today}}

\begin{{document}}
\maketitle

\begin{{abstract}}
{prelim}On MJD {date:.2f} the IceCube Neutrino Observatory  has reported the detection of a highly-energetic neutrino,
\textit{{{event}}}, with direction ra: {ra:.2f}$^\circ$, dec: {dec:.2f}$^\circ$ in equatorial and l: {l:.2f}$^\circ$, b:
{b:.2f}$^\circ$ in galactic coordinates. Due to the high-energy and the track-like signature the event has a good
pointing and is likely of astrophysical origin. With the goal of identifing
the corresponding electromagnetic counterpart we report here on the muli-wavenlength dissection of the region around the event.
\end{{abstract}}

\section{{Identifying Counterpart Candidates}}
The search for possible counterparts is based on the \textit{{VOU-Blazars}} tool \cite{{voublazar}} and follows closely the pipeline
developed in \cite{{Padovani:2018acg}}. \textit{{VOU-Blazars}} compares 32 multi-wavelength catalogs\footnote{{List of the 32 catalogs used in this
analysis: SDS82 , 3HSP , Fermi8YR ,1BIGB, MST9Y, FIRST, SUMSS, WGACAT, IPC2E, ZWCLUSTERS,
PSZ2, ABELL , SDSSWHL, CRATES, NVSS, SXPS, RASS, XMMSL, BMW , IPCSL, CHANDRA, MCXC,  5BZCat, SWXCS, PULSAR, F2PSR,
3FHL, 3FGL, 3XMM, MAXI, FermiMEV, AGILE}}
to find all positions in the vicinity of the alert with a \textit{{blazar-like}} emission profile in radio, optical and
X-ray. Each of these matches is then additionally checked against existing catalogs to identify the associated object if
possible. \\

In a second step a dedicated gamma-ray analysis is performed. In order to search for interesting emission features in the region we use \textit{{Fermi LAT}} data around the event time and run three different analysis
pipelines. Firstly, test-statistic maps are generated to search for unknown gamma-ray emmitters, e.g. indication for
gamma-ray emission from the previously identfied \textit{{VOU-Blazars}} source candidates. Subsequently SEDs and light curves are produced for each identified catalog
source and interesting \textit{{VOU-Blazars}} candidate. Based on these results we can finally search for specific
features in flux and spectral shape around the neutrino arrival time. \\

The gamma-ray analyses are based on the latest version of the FSSC Tools v11r5p3 and the fermipy package
\cite{{fermipy}}. We use the the standart procedures as described in the \textit{{Fermi LAT}} Cicerone \cite{{fermi_ci}}. The
dataset for the analysis contains events with photon energy above {emin} GeV in a time window between {mjd1:.1f} and {mjd2:.1f}.

\subsection{{Catalog Sources in the Region around {event}}}
The following list gives a quick overview about the catalog blazars in the region. Sources are sorted by their angular distance to the neutrino direction.
\\ \\
{cat_srcs}

\clearpage

\subsection{{Full Multi-Wavelength Study of the Region}}
In this section we present a full multi-wavelength search for possible neutrino counterparts. Starting from 32
multi-wavenlength catalogs the VOU-Blazar tool \cite{{voublazar}} uses the all the available radio, optical and X-ray
data in order to
identify blazar-like counterparts candidates. The full output of the tool can be found in the appendix.\\
The analysis pipeline consists of two parts: 1) The radio and x-ray data, as well as the resulting counterpart candidates are shown
and compared to the \textit{{Fermi LAT}} gamma-ray emission in the region around the neutrino alert 2) For all known blazars with a angular distance of less
than 1.5 degrees a multi-wavelength SED is constructed, including also a gamma-ray analysis that is started at the time of the
neutrino alert. For each source we also calculated a fixed-binning light curve. For all sources that have a previously
measured gamma-ray flux the bin-size is chosen in a way that each time bin has a ~5 sigma detection for the case of
steady emission. For all sources with unkown gamma-ray emission we choose a generic time windows of 200 days. The choice of this
time window is motivated a) by the expected faitness of the sources and b) the possible interested in time-dependent
neutrino follow-ups with IceCube. In the latter case, even for a realtively large neutrino flux, IceCube would need O(100 days) of
integration time in order to measure a reasonably significant signal.

\subsubsection{{Detailed description of the multi-wavelength SED}}
\label{{sec:sed}}
The multi-wavelength SED collects and visualizes all the available multi-wavelength data, as well as the result of the
gamma-ray analysis. The time evolution of the source is decoded in a color gradient from grey (old) to red (recent).
Here the grey SEDs point and bowties represent the \textit{{Fermi-LAT}} gamma-ray spectrum integrated over the entire
mission while the black SED points show the gamma-ray spectrum in a time window around the
neutrino arrival time. Colored bands indicate the corresponing spectral fits at different (if available) energy
thresholds if the significance of the measurement is above 3 $\sigma$. The green dashed and solid line show the sensitivity and discovery potential of the IceCube 7yr point-source analysis \cite{{psana}}, respectively.

\clearpage
\begin{{figure}}[h!]
\centering
\begin{{subfigure}}{{.5\textwidth}}
  \centering
  \includegraphics[width=.8\linewidth]{{{rx_map}}}
\end{{subfigure}}%
\begin{{subfigure}}{{.5\textwidth}}
\centering
  \includegraphics[width=.8\linewidth]{{{vou_pic}}}
\end{{subfigure}}
 \caption{{Left Plot: Radio and X-ray sources within 120 arc-minutes of the position of the neutrino event. Symbol
diameters are proportional to source intensity. Radio sources appear as red filled circles, X-ray sources as open blue
circles  Right Plot: Counterpart candidates in a 120 arc-minutes radius around the event direction. Dark blue circles represent LBL type candidates, that is sources with flux ratio in the range observed in the sample of LBL blazars of
the latest edition of the BZCAT catalogue \cite{{Massaro:2015nia}}, cyan symbols are for IBL type candidates, and orange symbols are for
HBL candidates. Known blazars are marked by a red diamond if they are included in the BZCAT catalogue or a star if
they are part of the 2WHSP sample. The shaded area marks a circle of 90 arcmis around the events-best fit direction. }}
\end{{figure}}


\begin{{figure}}[h!]
\centering
\begin{{subfigure}}{{.37\textwidth}}
  \includegraphics[height=7cm]{{{ts_map_short}}}

\end{{subfigure}}%
\begin{{subfigure}}{{.63\textwidth}}
\centering
  \includegraphics[height=7cm]{{{ts_map}}}
\end{{subfigure}}
\caption{{Left: The TS map of the region for a time window of 200 days (MJD {tsmjd1:.1f}, {tsmjd2:.1f}) around the
neutrino arrival time. Right: The TS map of
the region for the entire Fermi LAT mission (MJD {mjd1:.1f} to {mjd2:.1f}). Both plots are done for energies above
{tsemin} GeV. The analysis model includes all known sources from the 3FGL catalog in a region of 8 degrees around the
center. The blue circle indicates a radius of 90 arcmin around the the event best-fit direction. }}
\end{{figure}}

\clearpage
\section{{SEDs and Light Curves}}
{src_latex}

\clearpage
\section{{Appendix}}
{vou_output}


{{\small\bibliographystyle{{unsrt}}
\bibliography{{sample}}}}
\end{{document}}
