\documentclass[a4paper]{{article}}
\usepackage[numbers]{{natbib}}
\usepackage[english]{{babel}}
\usepackage[utf8]{{inputenc}}
\usepackage{{amsmath}}
\usepackage{{graphicx}}
\usepackage{{xcolor}}
\usepackage[outdir=./converted]{{epstopdf}}
\usepackage[left=30mm, right=30mm, bottom=30mm]{{geometry}}
\title{{Dissecting the Region around \textit{{{event}}}}}

\author{{Authors}}

\date{{\today}}

\begin{{document}}
\maketitle

\begin{{abstract}}
{prelim}On MJD {date} the IceCube Neutrino Observatory  has reported the detection of a highly-energetic neutrino, \textit{{{event}}}, with direction
ra: {ra:.2f}$^\circ$, dec: {dec:.2f}$^\circ$ (l: {l:.2f}$^\circ$, b: {b:.2f}$^\circ$) that has a good chance of being of astropysical origin. With the goal of identifing
the corresponding electromagnetic counterpart we report here on the muli-wavenlength dissection of the region around the event.
\end{{abstract}}

\section{{Identifying counterpart candidates}}
The search for possible counterparts is based on the VOU-Blazar Tool \cite{{voublazar}} and follows closely the pipeline
developed in \cite{{Padovani:2018acg}}.  VOU-Blazar compares 28
multi-wavelengths catalogs to find all positions in the vicinity of the event that have a \textit{{blazar-like}} radio
to X-ray emission ratio. For each match, existing source catalogs are checked to identify possible counterparts. \\

Subsequently we run a Fermi-LAT gamma-ray analysis based on the latest version of the FSSC Tools v11r5p3 and the fermipy package
\cite{{fermipy}} and using the standart \\textit{{gtlike}} procedure. The dataset contains photons with energies above
{emin} GeV in a time window between {mjd1:.1f} and {mjd2:.1f}

\subsection{{Quick summary: Catalog sources in the region}}
The following list gives a short overview about the most interesting catalog sources in the region. Check the following section for a more detailed analysis. The sources are sorted by their distance to the neutrino direction.
\\ \\
{cat_srcs}

\subsection{{Full multi-wavelength study of the region}}
In this section the full multi-wavelength search for neutrino counterparts is presented. Despite catalog sources also candidate sources are listed based on the x-ray to radio emission. The sources are sorted by their right ascension.
\\ \\ \\
{vou_output}

\clearpage
\begin{{figure}}[h!]
\centering
  \includegraphics[width=0.5\textwidth]{{{rx_map}}}
  \caption{{Radio and X-ray sources within 120 arc-minutes of the position of the neutrino event. Symbol diameters are
proportional to source intensity. Radio sources appear as red filled circles, X-ray sources as open blue circles}}
\end{{figure}}


\begin{{figure}}[h!]
\centering
  \includegraphics[width=0.5\textwidth]{{{vou_pic}}}
  \caption{{Counterpart candidates in a 120 arc-minutes radius around the event direction. Dark blue circles represent LBL type candidates, that is sources with flux ratio in the range observed in the sample of LBL blazars of
the latest edition of the BZCAT catalogue (Massaro et al. 2015), cyan symbols are for IBL type candidates, and orange symbols are for
HBL candidates. Known blazars are also marked by a red diamond if they are included in the BZCAT catalogue or a star if
they are part of the 2WHSP sample. The diameters of filled and open circles are proportional to radio flux density and X-ray flux, respectively. }}
\end{{figure}}

\clearpage

\begin{{figure}}[h!]
\centering
  \includegraphics[width=0.9\textwidth]{{{ts_map}}}
  \caption{{The TS Map of the region for MJD {mjd1:.1f} and {mjd2:.1f} and energies above {energy} GeV. The analysis model includes all the known sources from the 3FGL Catalog}}
\end{{figure}}

\begin{{figure}}[h!]
\centering
  \includegraphics[width=0.9\textwidth]{{{res_map}}}
  \caption{{The Residual Map of the region for MJD {mjd1:.1f} and {mjd2:.1f} and energies above {energy} GeV. The analysis model includes all the known sources from the 3FGL Catalog}}
\end{{figure}}

\clearpage
\section{{The closer look: SEDs and light curves}}
{src_latex}

{{\small\bibliographystyle{{unsrt}}
\bibliography{{sample}}}}
\end{{document}}
